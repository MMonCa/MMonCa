\section{Citing}

Please, if using \MMonCa\ cite it as in Ref.~\cite{MARTIN-BRAGADO-CPC13}.
If you use the LKMC module only, you can cite Ref.~\cite{MARTIN-BRAGADO-APL11}.
If using the BCA software (Ion Implant Simulator) cite
Ref.~\cite{HERNANDEZ-MANGAS-JAP02}


\section{Introduction}
\subsection{Installation from source code}
\begin{enumerate}
 \item Download and unpack the \MMonCa\ source code from the IMDEA Materials web page. {\tt http://www.materials.imdea.org/MMonCa}.
 \item Be sure you have a modern version of the {\tt g++} compiler installed in your system. 
\begin{itemize}
\item Type \param{make} to obtain some help on how the \param{Makefile} system works.
\item The make command expects one \param{machine} to be specified. The available machines are in the directory \param{sysmakes}.
\item Each \param{machine} is a specific configuration for a particular architecture. The \param{make} commands creates a binary directory called {\tt Obj\verb+_+machine} directory, with the \param{mmonca} binary inside.
\end{itemize}
\item The compilation system looks for several libraries. Some of them are compulsory, other optional if you want to have the \param{iris} FEM module running. In an Ubuntu or Debian system these libraries can be installed using \param{apt-get}. The required libraries (for Ubuntu) are:
\begin{description}
\item[compulsory] {\tt tcl-dev}.
\item[compulsory] {\tt libboost-dev}.
\item[compulsory] {\tt libboost-iostreams-dev}
\item[iris] {\tt libsuperlu3-dev}.
\item[iris] {\tt liblapack-dev}.
\end{description}
 \item Depending on the location of your workspace, add the following line to {\tt .bashrc: export MCPATH=/home/\textit{username}/workspace/MMonCa/config}
 \item After compiling the code navigate to ({\tt cd} command): \\
{\tt /home/\textit{username}/workspace/MMonCa/test}, and enter {\tt runAll.sh} in order to run the KMC tests and confirm a successful installation. The test suite expects a MMonCa binary in a directory called Release. You can copy or link a binary there for the test suite to work. By default, it uses the binary inside \verb-Obj_g++-.
\end{enumerate}



\subsection{Submission Policy}
\index{submission}
When collaborating on a program, it is essential to preserve the functionality of existing code before adding your own. Therefore, after code is developed it is important to run all existing \param{test} to check that there are no failures. Also, new tests are required to be created for every new feature in the code.

The test-suite can be run with the command \param{runAll.sh} in the \param{test} directory as previously explained.

\subsection{Installation from binary sources}
\begin{itemize}
\item Untar and unzip the \MMonCa\ binary with something like:
\begin{verbatim}
tar -xvzf MMonCa-VERSION-bin.tar.gz
\end{verbatim}
\item Be sure you have the following libraries installed in your system:
\begin{itemize}
\item libtcl
\item libblas
\item liblapack
\item libsuperlu
\end{itemize}
 \item Depending on the location of your workspace, add the following line to {\tt .bashrc: export MCPATH=/home/\textit{username}/workspace/MMonCa/config}.\index{MCPATH}
\end{itemize}
