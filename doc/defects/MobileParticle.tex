\subsection{\tt MobileParticle}
\index{MobileParticle}
MobileParticles are single \idx{point defects}, i.e., they are represented with three coordinates only. They can be either self point defects (interstitials or vacancies) or impurities (carbon, helium, etc). For each impurity or element $A$, a mobile particle with each of the 5 different positions explained in section~\ref{sec:position} can also be found. 

MobileParticle parameters, for a given species $A$ are the following ones:
\begin{description}
\item [A(migration)] An arrhenius with the \idx{diffusivity}.
\item [A(formation)] An arrhenius with the  \idx{formation energy} in the bulk.
\end{description}

The events that a MobileParticle can perform are:
\begin{description}
\item[\idx{Migration}] Standard.
\item[\idx{Break-up}] For instance, $AI \rightarrow AM + MI$, being $M$ the material. The formation parameters are used to define the break-up frequencies, but $E_\mathrm{binding}(AI) = E_f(AM) + E_f(MI) - E_f(AI)$.
\item[\idx{Frank-Turnbull}] For instance, $AM  \rightarrow AI +VM$ or $AM \rightarrow AV + MI$. Formation values are used to compute such break-up frequencies.
\end{description}

If the \idx{charge model} is activated, additional state charges can be defined for each particle. For instance

\begin{description}
\item [AM(state.charge)] AM 0
\item [AI(state.charge)] \verb+AI_0+ 0 \verb+AI_-+ -1
\end{description}

The \idx{levels} in the \idx{band gap} for the transitions between charges have also to be defined.

\begin{lstlisting}
float AI(e(-1,0))   .5
\end{lstlisting}

And, finally, a frequency to \idx{update} the charge state between different states is required.

\begin{lstlisting}
arrhenius AI(update) { 1 0.7 }
\end{lstlisting}

